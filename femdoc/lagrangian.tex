
%------------------------------------------------------------------------
%
%    Copyright (C) 1985-2020  Georg Umgiesser
%
%    This file is part of SHYFEM.
%
%    SHYFEM is free software: you can redistribute it and/or modify
%    it under the terms of the GNU General Public License as published by
%    the Free Software Foundation, either version 3 of the License, or
%    (at your option) any later version.
%
%    SHYFEM is distributed in the hope that it will be useful,
%    but WITHOUT ANY WARRANTY; without even the implied warranty of
%    MERCHANTABILITY or FITNESS FOR A PARTICULAR PURPOSE. See the
%    GNU General Public License for more details.
%
%    You should have received a copy of the GNU General Public License
%    along with SHYFEM. Please see the file COPYING in the main directory.
%    If not, see <http://www.gnu.org/licenses/>.
%
%    Contributions to this file can be found below in the revision log.
%
%------------------------------------------------------------------------

Lagrangian analysis provides a powerful tool to evaluate the output
of ocean circulation models. SHYFEM is equipped with a 3-D 
particle-tracking module, which simulates the trajectory of 
particles as a function of the hydrodynamics. 

The vertical components of the turbulent diffusion velocity is
computed using the Milstein scheme reported by Grawe 
\cite{Grawe2010}. The horizontal diffusion was computed using a 
random walk technique based on Fisher \cite{Fisher1979}, with the 
turbulent diffusion coefficients obtained by means of the 
Smagorinsky \cite{Smagorinsky1993} formulation. The wind drag 
and Stokes drift contribution to the total transport is parametrized
by |stkpar| factor. An additional calibration parameter to account
for the drifter inertia could be set (|dripar|). The model allows
particle to beach on the shore (|lbeach|). 

The particle-tracking model can be also used off-line (parameter
|idtoff|). In this case it uses the Eulerian hydrodynamic fields 
generated by the forecast system. The main advantage of the 
off-line approach is that the trajectory calculation typically 
takes much less computational effort than the driving hydrodynamic 
model.

The lagrangian particles can be released:
\begin{itemize}
\item inside the given areas (filename |lgrlin|). If this file is not 
      specified they are released over the whole domain. The amount of
      particles released and the time step is specified by |nbdy| and 
      |idtl|.
\item at selected times and location, e.g. along a drifter track
      (filename |lgrtrj|). |nbdy| particles are released at the times
      and location specified in the file.
\item as initial particle distribution (filename |lgrini|) at time
      |itlgin|. This file has the same format as the lagrangian output.
\item at the open boundaries, either as particles per second or per
      volume flux (parameter |lgrpps|).
\end{itemize}

The particle-tracking model is activated by setting |ilagr| > 0.
The lagrangian module runs between the times |itlanf| and |itlend|.
See more details in the list of parameters and they description 
reported in the appendix.

The lagrangian model can be used in other sub-modules to specifically
simulate sediments (|ised=1|), oil (|ioil=1|) and larvae (|ilarv=1|).

