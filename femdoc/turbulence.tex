
%------------------------------------------------------------------------
%
%    Copyright (C) 1985-2018  Georg Umgiesser
%
%    This file is part of SHYFEM.
%
%    SHYFEM is free software: you can redistribute it and/or modify
%    it under the terms of the GNU General Public License as published by
%    the Free Software Foundation, either version 3 of the License, or
%    (at your option) any later version.
%
%    SHYFEM is distributed in the hope that it will be useful,
%    but WITHOUT ANY WARRANTY; without even the implied warranty of
%    MERCHANTABILITY or FITNESS FOR A PARTICULAR PURPOSE. See the
%    GNU General Public License for more details.
%
%    You should have received a copy of the GNU General Public License
%    along with SHYFEM. Please see the file COPYING in the main directory.
%    If not, see <http://www.gnu.org/licenses/>.
%
%    Contributions to this file can be found below in the revision log.
%
%------------------------------------------------------------------------

In the Reynolds equations turbulent eddy diffusivities and viscosities are
introduced into the equations that must be parameterized and given some
value. Moreover SHYFEM assumes the hydrostatic approximation. Therefore,
there is the need to parameterize the non-hydrostatic effects. These
are considered sub-scale processes which are mainly of convective nature.

Vertical eddy viscosities and diffusivities have to be defined if
there is the intent to model the turbulence effects.  These vertical
eddy viscosities and diffusivities can be set to constant values,
defining |vistur| and |diftur| in the |$para| section.  There is also
the opportunity to compute, at each timestep, variable values of them,
using the turbulence closure module.

The parameter that has to be set in order too choose the turbulence
scheme is |iturb| in the |$para| section..  If |iturb|=0 the vertical
eddy viscosity and eddy diffusivity are set constant (default 0) and
must be defined in |vistur| and |diftur|.

If |iturb|=1 the GOTM turbulence closure module is used.
If |iturb|=2 the turbulence closure scheme applied is the $k-\epsilon$ model. 
Finally, if |iturb|=3, the Munk-Anderson model is used.

With |iturb|=1, the file |gotmturb.nml| must be provided that sets all
necessary parameters.  This file must be declared in the section |$name|
for the item |gotmpa|.

A default |gotmturb.nml| file is provided and it allows the
computation of the vertical eddy viscosity and eddy diffusivity by
means of the GOTM  $k-\epsilon$ model.  More information on the
GOTM turbulence closure module can be found in the GOTM Manual
\footnote{http://www.gotm.net/index.php?go=documentation}.

If the turbulence module should be used, a value of |iturb|=1 is
recommended.  An example of the settings for the turbulence closure
scheme is given in \Fig\figref{turbulence}.

\begin{figure}[ht]
\begin{verbatim}
$para
        iturb = 1
$end
$name
        gotmpa = 'gotmturb.nml'
$end
\end{verbatim}
\caption{Example of turbulence settings. The GOTM module for the
turbulence closure is used. The parameters are contained in file
gotmturb.nml.}
\label{fig:turbulence}
\end{figure}

