
%------------------------------------------------------------------------
%
%    Copyright (C) 1985-2018  Georg Umgiesser
%
%    This file is part of SHYFEM.
%
%    SHYFEM is free software: you can redistribute it and/or modify
%    it under the terms of the GNU General Public License as published by
%    the Free Software Foundation, either version 3 of the License, or
%    (at your option) any later version.
%
%    SHYFEM is distributed in the hope that it will be useful,
%    but WITHOUT ANY WARRANTY; without even the implied warranty of
%    MERCHANTABILITY or FITNESS FOR A PARTICULAR PURPOSE. See the
%    GNU General Public License for more details.
%
%    You should have received a copy of the GNU General Public License
%    along with SHYFEM. Please see the file COPYING in the main directory.
%    If not, see <http://www.gnu.org/licenses/>.
%
%    Contributions to this file can be found below in the revision log.
%
%------------------------------------------------------------------------

%%%%%%%%%%%%%%%%%%%%%%%%%%%%%%%%%%%%%%%%%%%%%%%%%%%%%%%%%%
%%%%%%% user commands %%%%%%%%%%%%%%%%%%%%%%%%%%%%%%%%%%%%
%%%%%%%%%%%%%%%%%%%%%%%%%%%%%%%%%%%%%%%%%%%%%%%%%%%%%%%%%%

\newcommand{\paren}[1]	{ \left( #1 \right) }
\newcommand{\mez}{\mbox{$\frac{1}{2}$}}
\newcommand{\dpp}{\mbox{$\partial$}}
\newcommand{\zz}{\zeta}
\newcommand{\un}{\mbox{$U^{n+1}$}}
\newcommand{\uo}{\mbox{$U^{n}$}}
\newcommand{\vn}{\mbox{$V^{n+1}$}}
\newcommand{\vo}{\mbox{$V^{n}$}}
\newcommand{\zn}{\mbox{$\zz^{n+1}$}}
\newcommand{\zo}{\mbox{$\zz^{n}$}}
\newcommand{\up}{\mbox{$U^{\prime}$}}
\newcommand{\vp}{\mbox{$V^{\prime}$}}
\newcommand{\zp}{\mbox{$\zz^{\prime}$}}
\newcommand{\dzxp}{\mez \frac{\dpp (\zn + \zo)}{\dpp x}}
\newcommand{\dzyp}{\mez \frac{\dpp (\zn + \zo)}{\dpp y}}
\newcommand{\dzxn}{\mbox{$\frac{\dpp \zn}{\dpp x}$}}
\newcommand{\dzyn}{\mbox{$\frac{\dpp \zn}{\dpp y}$}}
\newcommand{\dzxo}{\mbox{$\frac{\dpp \zo}{\dpp x}$}}
\newcommand{\dzyo}{\mbox{$\frac{\dpp \zo}{\dpp y}$}}

\newcommand{\tdif}[1] {\frac{\partial #1}{\partial t}}
\newcommand{\xdif}[1] {\frac{\partial #1}{\partial x}}
\newcommand{\ydif}[1] {\frac{\partial #1}{\partial y}}
\newcommand{\zdif}[1] {\frac{\partial #1}{\partial z}}
\newcommand{\dt} {\mbox{$\Delta t$}}
\newcommand{\dthalf} {\mbox{$\frac{\Delta t}{2}$}}
\newcommand{\dtt} {\mbox{$\frac{\Delta t}{2}$}}
\newcommand{\dx} {\mbox{$\Delta x$}}
\newcommand{\dy} {\mbox{$\Delta y$}}

\newcommand{\beq} {\begin{equation}}
\newcommand{\eeq} {\end{equation}}
\newcommand{\beqa} {\begin{eqnarray}}
\newcommand{\eeqa} {\end{eqnarray}}

\newcommand{\olds} {\mbox{$\scriptstyle (0)$}}
\newcommand{\news} {\mbox{$\scriptstyle (1)$}}
\newcommand{\meds} {\mbox{$\scriptscriptstyle (\frac{1}{2})$}}
\newcommand{\half} {\mbox{$\scriptstyle \frac{1}{2}$}}

\newcommand{\nsz} {\normalsize}
\newcommand{\uold} {\mbox{$U^{\olds}$}}
\newcommand{\vold} {\mbox{$V^{\olds}$}}
\newcommand{\unew} {\mbox{$U^{\news}$}}
\newcommand{\vnew} {\mbox{$V^{\news}$}}
\newcommand{\zold} {\zeta^{(0)}}
\newcommand{\znew} {\zeta^{(1)}}
\newcommand{\resr} {{\cal R}}
\newcommand{\drho} {\frac{1}{\rho_{0}}}
\newcommand{\fracs}[2] {\mbox{$\frac{#1}{#2}$}}
%\newcommand{\deltat} {\mbox{$\tilde{\delta}$}}
%\newcommand{\gammat} {\mbox{$\tilde{\gamma}$}}
\newcommand{\ffxx} {\tilde{f_x}}
\newcommand{\ffyy} {\tilde{f_y}}

\newcommand{\uv} {{\bf U}}
\newcommand{\uvold} {{\bf U^{(0)}}}
\newcommand{\uvnew} {{\bf U^{(1)}}}
\newcommand{\af} {\alpha_{f}}
\newcommand{\ac} {\alpha_{c}}
\newcommand{\am} {\alpha_{m}}
\newcommand{\duv} {\Delta {\bf U}}
\newcommand{\dzeta} {\Delta \zeta}
\newcommand{\iv} {{\bf I}}
\newcommand{\ivh} {\hat{\bf I}}
\newcommand{\fv} {{\bf F}}
\newcommand{\uvh} {\hat{\bf U}}


%%%%%%%%%%%%%%%%%%%%%%%%%%%%%%%%%%%%%%%%%%%%%%%%%%%%%%%%%%
%%%%%%% hyphenation %%%%%%%%%%%%%%%%%%%%%%%%%%%%%%%%%%%%%%
%%%%%%%%%%%%%%%%%%%%%%%%%%%%%%%%%%%%%%%%%%%%%%%%%%%%%%%%%%


\section{Equations and Boundary Conditions}

The equations used in the model are the well known vertically integrated
shallow water equations in their formulation with water levels and
transports.

\beq \label{ubar}
\tdif{U} + gH \xdif{\zeta} + RU + X = 0
\eeq
\beq
\tdif{V} + gH \ydif{\zeta} + RV + Y = 0
\eeq
\beq \label{zcon}
\tdif{\zeta} + \xdif{U} + \ydif{V} = 0
\eeq
where $\zeta$ is the water level, $u,v$ the velocities in $x$ and $y$
direction,
$U,V$ the vertical integrated velocities (total  or barotropic
transports)
\[
 U = \int_{-h}^{\zeta} u \: dz \; \hspace{1.cm}
 V = \int_{-h}^{\zeta} v \: dz \;
\]
$g$ the gravitational acceleration, $H=h+\zeta$ the total water
depth, $h$ the undisturbed water depth,
$t$ the time and $R$ the friction coefficient. The terms $X,Y$ contain
all other terms that may be added to the equations like the wind stress or
the nonlinear terms and that need not be treated implicitly in the
time discretization.
following treatment.

The friction coefficient has been expressed as
\begin{equation}
	R = \frac{g \sqrt{u^{2}+v^{2}}}{C^{2} H}
\end{equation}
with $C$ the Chezy coefficient. The Chezy term is itself not retained
constant but varies with the water depth as
\begin{equation}
	C = k_{s} H^{1/6}
\end{equation}
where $k_{s}$ is the Strickler coefficient.

In this version of the model the Coriolis term, the turbulent friction term
and the nonlinear advective terms have not been implemented.

At open boundaries the water levels are prescribed. At closed boundaries
the normal velocity component is set to zero whereas the tangential velocity
is a free parameter. This corresponds to a full slip condition.


%%%%%%%%%%%%%%%%%%%%%%%%%%%%%%%%%%%%%%%%%%%%%%%%%%%%%%%%%%%%%%%%%%%%%%%%%
%%%%%%%%%%%%%%%%%%%%%%%%%%%%%%%%%%%%%%%%%%%%%%%%%%%%%%%%%%%%%%%%%%%%%%%%%
%%%%%%%%%%%%%%%%%%%%%%%%%%%%%%%%%%%%%%%%%%%%%%%%%%%%%%%%%%%%%%%%%%%%%%%%%
%%%%%%%%%%%%%%%%%%%%%%%%%%%%%%%%%%%%%%%%%%%%%%%%%%%%%%%%%%%%%%%%%%%%%%%%%
%%%%%%%%%%%%%%%%%%%%%%%%%%%%%%%%%%%%%%%%%%%%%%%%%%%%%%%%%%%%%%%%%%%%%%%%%





\section{The Model}

The model uses the semi-implicit time discretization to accomplish
the time integration. In the space the finite element method has
been used, not in its standard formulation, but using staggered finite
elements. In the following a description of the method is given.



%%%%%%%%%%%%%%%%%%%%%%%%%%%%%%%%%%%%%%%%%%%%%%%%%%%%%%%%%%%%%%%%%%%%%%%%%
%%%%%%%%%%%%%%%%%%%%%%%%%%%%%%%%%%%%%%%%%%%%%%%%%%%%%%%%%%%%%%%%%%%%%%%%%
%%%%%%%%%%%%%%%%%%%%%%%%%%%%%%%%%%%%%%%%%%%%%%%%%%%%%%%%%%%%%%%%%%%%%%%%%
%%%%%%%%%%%%%%%%%%%%%%%%%%%%%%%%%%%%%%%%%%%%%%%%%%%%%%%%%%%%%%%%%%%%%%%%%
%%%%%%%%%%%%%%%%%%%%%%%%%%%%%%%%%%%%%%%%%%%%%%%%%%%%%%%%%%%%%%%%%%%%%%%%%



\subsection{Discretization in Time - The Semi-Implicit Method}

Looking for an efficient time integration method
a semi-implicit scheme has been chosen.
The semi-implicit scheme combines the advantages of
the explicit and the implicit scheme. It is unconditionally stable for any
time step $\dt$ chosen and allows the two momentum equations to be
solved explicitly without solving a linear system. 

The only equation
that has to be solved implicitly is the continuity equation. Compared
to a fully implicit solution of the shallow water equations the dimensions
of the matrix are reduced to one third. Since the solution of a linear
system is roughly proportional to the cube of the dimension of the system
the saving in computing time is approximately a factor of 30.

It has to be pointed out that it is important not to be limited with the time
step by the CFL criterion for the speed of the external gravity waves
\[
        \dt < \frac{\dx}{\sqrt{gH}}
\]
where $\dx$ is the minimum distance between the nodes in an element.
With the discretization described below in most parts of the lagoon
we have $\dx \approx$ 500m and $H \approx$ 1m, so $\dt \approx 200$ sec.
But the limitation of the time step is determined by the worst case.
For example, for $\dx = 100$ m and $H = 40$ m
the time step criterion would be $\dt < 5$ sec, a
prohibitive small value.

The equations (1)-(3) are discretized as follows
\begin{equation}
\label{zn}
\frac{\zn-\zo}{\dt}
                        + \mez \frac{\dpp (\un + \uo)}{\dpp x}
                        + \mez \frac{\dpp (\vn + \vo)}{\dpp y} = 0
\end{equation}
\begin{equation}
\frac{\un-\uo}{\dt} + gH \dzxp + R \un + X = 0
\end{equation}
\begin{equation}
\frac{\vn-\vo}{\dt} + gH \dzyp + R \vn + Y = 0
\end{equation}

With this time discretization the friction term has been formulated
fully implicit, $X,Y$ fully explicit and all the other terms
have been centered in time. The reason for the implicit treatment
of the friction term is to avoid a sign inversion in the term when
the friction parameter gets too high. An example of this behavior is
given in Backhaus \cite{Backhaus83}.

If the two momentum equations are solved for the unknowns $\un$ and $\vn$
we have
\begin{equation}
\label{un}
\un = \frac{1}{1+\dt R} \paren{ \uo - \dt gH \dzxp - \dt X }
\end{equation}
\begin{equation}
\label{vn}
\vn = \frac{1}{1+\dt R} \paren{ \vo - \dt gH \dzyp - \dt Y }
\end{equation}

If $\zn$ were known, the solution for
$\un$ and $\vn$ could directly be given. To find $\zn$ we insert
(\ref{un}) and (\ref{vn}) in (\ref{zn}). After some transformations
(\ref{zn}) reads
\begin{eqnarray} \label{zsys}
        \zn
    & - &
        (\dt/2)^{2} \frac{g}{1+\dt R}         \nonumber
	\paren{ \frac{\dpp}{\dpp x}(H \dzxn) + \frac{\dpp}{\dpp y}(H \dzyn) } \\
    & = &
        \zo + (\dt/2)^{2} \frac{g}{1+\dt R}
	\paren{ \frac{\dpp}{\dpp x}(H \dzxo) + \frac{\dpp}{\dpp y}(H \dzyo) } \\
    & - & (\dt/2) \paren{ \frac{2+\dt R}{1+\dt R} }
        \paren{                               \nonumber
          \frac{\dpp \uo}{\dpp x}
        + \frac{\dpp \vo}{\dpp y}
        } \\
    & + & \frac{\dt^{2}}{2(1+\dt R)}            \nonumber
                \paren{ \frac{\dpp X}{\dpp x} + \frac{\dpp Y}{\dpp y} }
\end{eqnarray}

The terms on the left hand side contain the unknown $\zn$, the right hand
contains only known values of the old time level. If the spatial derivatives
are now expressed by the finite element method a linear system with the unknown
$\zn$ is obtained and can be solved by standard methods. Once the solution
for $\zn$ is obtained it can be substituted into (\ref{un}) and (\ref{vn})
and these two equations can be solved explicitly. In this way all unknowns
of the new time step have been found.

Note that the variable $H$ also contains the water level through
$H=h+\zz$. In order to avoid the equations to become nonlinear $\zz$
is evaluated at the old time level so $H=h+\zo$ and $H$ is a known quantity.


%%%%%%%%%%%%%%%%%%%%%%%%%%%%%%%%%%%%%%%%%%%%%%%%%%%%%%%%%%%%%%%%%%%%%%%%%
%%%%%%%%%%%%%%%%%%%%%%%%%%%%%%%%%%%%%%%%%%%%%%%%%%%%%%%%%%%%%%%%%%%%%%%%%
%%%%%%%%%%%%%%%%%%%%%%%%%%%%%%%%%%%%%%%%%%%%%%%%%%%%%%%%%%%%%%%%%%%%%%%%%
%%%%%%%%%%%%%%%%%%%%%%%%%%%%%%%%%%%%%%%%%%%%%%%%%%%%%%%%%%%%%%%%%%%%%%%%%
%%%%%%%%%%%%%%%%%%%%%%%%%%%%%%%%%%%%%%%%%%%%%%%%%%%%%%%%%%%%%%%%%%%%%%%%%




%%%%%%%%%%%%%%%%%%%%%%%%%%%%%%%%%%%%%%%%%%%%%%%%%%%%%%%%%%%%%%%%%%%%%%%%%
%%%%%%%%%%%%%%%%%%%%%%%%%%%%%%%%%%%%%%%%%%%%%%%%%%%%%%%%%%%%%%%%%%%%%%%%%
%%%%%%%%%%%%%%%%%%%%%%%%%%%%%%%%%%%%%%%%%%%%%%%%%%%%%%%%%%%%%%%%%%%%%%%%%
%%%%%%%%%%%%%%%%%%%%%%%%%%%%%%%%%%%%%%%%%%%%%%%%%%%%%%%%%%%%%%%%%%%%%%%%%
%%%%%%%%%%%%%%%%%%%%%%%%%%%%%%%%%%%%%%%%%%%%%%%%%%%%%%%%%%%%%%%%%%%%%%%%%



\subsection{Discretization in Space - The Finite Element Method}


While the time discretization has been explained above, the discretization
in space has still to be carried out. This is done 
using staggered finite elements. 
With the semi-implicit method described above
it is shown below that using linear triangular elements
for all unknowns 
will not be mass conserving. Furthermore the resulting model
will have propagation properties that introduce high numeric damping
in the solution of the equations.

For these reasons a quite new approach has been adopted here. The water
levels and the velocities (transports) are described by using form
functions of different order, being the standard linear form functions
for the water levels but stepwise constant form functions for the
transports. This will result in a grid that resembles more a staggered
grid in finite difference discretizations.

\subsubsection{Formalism}

Let $u$ be an approximate solution of a linear differential
equation $L$. We expand $u$ with the help of basis functions $\phi_{m}$
as
\begin{equation}
\label{exp}
	u=\phi_{m} u_{m} \mbox{\hspace{1cm}} m=1,K
\end{equation}
where $u_{m}$ is the coefficient of the function $\phi_{m}$ and $K$
is the order of the approximation.
In case of linear finite
elements it will just be the number of nodes of the grid used to
discretize the domain.

To find the values $u_{m}$ we try to minimize the residual
that arises when $u$ is introduced into $L$ multiplying the equation $L$
by some weighting functions $\Psi_{n}$ and
integrating over the whole domain leading to
\begin{equation}
\label{int}
\int_{\Omega} \psi_{n} L(u) \: d\Omega \; = 
\int_{\Omega} \psi_{n} L(\phi_{m} u_{m}) \: d\Omega \;
= u_{m} \int_{\Omega} \psi_{n} L(\phi_{m}) \: d\Omega \;
\end{equation}

If the integral is identified with the elements of a matrix $a_{nm}$
we can write (\ref{int}) also as a linear system
\begin{equation} \label{sys}
	a_{nm}u_{m} = 0 \mbox{\hspace{1cm}} n=1,K \hspace{0.5cm} m=1,K
\end{equation}

Once the basis and weighting functions have been specified the system
may be set up and (\ref{sys}) may be solved for the unknowns $u_{m}$.






\subsubsection{Staggered Finite Elements}

For decades finite elements have been used in fluid mechanics in
a standardized manner.
The form functions $\phi_{m}$ were chosen as continuous piecewise linear
functions allowing a subdivision of the whole area of interest into small
triangular elements specifying the coefficients $u_{m}$ at the vertices
(called nodes)
of the triangles. The functions $\phi_{m}$ are 1 at node
$m$ and 0 at all other nodes and thus different from 0 only in the
triangles containing the node $m$.
An example is given in the upper left part of Fig. 1a
where the form function for node $i$ is shown. The full circle indicates
the node where the function $\phi_{i}$ take the value
1 and the hollow circles where they are 0.


\begin{figure}
\vspace{5.cm}
\caption{a) form functions in domain \hspace{1.cm} b) domain of
influence of node $i$}
%\end{figure}

\begin{picture}(300,10)

\put(10,30){
\begin{picture}(1,1)
\put(80,90){\line(0,1){30}}
\put(80,90){\line(3,1){30}}
\put(80,90){\line(2,-3){20}}
\put(80,90){\line(-2,-3){20}}
\put(80,90){\line(-3,1){30}}
\put(60,60){\line(1,0){40}}
\put(60,60){\line(1,-2){20}}
\put(60,60){\line(-2,-3){20}}
\put(60,60){\line(-4,1){40}}
\put(60,60){\line(-1,4){10}}
\put(50,100){\line(-1,-1){30}}
\put(50,100){\line(-3,1){30}}
\put(50,100){\line(-1,4){10}}
\put(50,100){\line(3,2){30}}
\put(80,120){\line(-2,1){40}}
\put(80,120){\line(1,3){10}}
\put(80,120){\line(3,-2){30}}
\put(110,100){\line(-2,5){20}}
\put(110,100){\line(1,1){30}}
\put(110,100){\line(1,-1){30}}
\put(110,100){\line(-1,-4){10}}
\put(100,60){\line(4,1){40}}
\put(100,60){\line(3,-4){30}}
\put(100,60){\line(-1,-2){20}}
\put(130,20){\line(1,5){10}}
\put(130,20){\line(-1,0){50}}
\put(140,130){\line(0,-1){60}}
\put(140,130){\line(-5,2){50}}
\put(40,140){\line(5,1){50}}
\put(40,140){\line(-2,-3){20}}
\put(20,110){\line(0,-4){40}}
\put(40,30){\line(-1,2){20}}
\put(40,30){\line(4,-1){40}}
\thicklines
\put(50,120){\line(1,-1){30}}
\put(50,120){\line(1,0){30}}
\put(50,120){\line(-1,2){10}}
\put(50,120){\line(-3,-1){30}}
\put(50,120){\line(-3,-5){30}}
\put(50,120){\line(1,-6){10}}
\put(50,120){\line(0,-1){20}}
\put(100,80){\line(-1,-2){20}}
\put(100,80){\line(3,-4){30}}
\put(100,80){\line(0,-2){20}}
\put(80,40){\line(0,-2){20}}
\put(80,40){\line(1,0){50}}
\put(130,40){\line(0,-2){20}}
\thinlines
\put(100,30){$n$}
\put(45,88){$i$}
\put(5,90){$\phi_{i}$}
\put(100,10){$\psi_{n}$}
%\put(80,90){\circle*{5}}
%\put(60,60){\circle*{5}}
\put(50,100){\circle*{5}}
%\put(80,120){\circle*{5}}
%\put(110,100){\circle*{5}}
%\put(100,60){\circle*{5}}
\put(100,60){\circle*{5}}
\put(80,20){\circle*{5}}
\put(130,20){\circle*{5}}
%\put(50,100){\circle{7}}
\put(80,90){\circle{7}}
\put(80,120){\circle{7}}
\put(60,60){\circle{7}}
\put(20,70){\circle{7}}
\put(20,110){\circle{7}}
\put(40,140){\circle{7}}
\end{picture}}
%\end{center}



%\put(250,30){
\put(200,30){
\begin{picture}(1,1)
\put(80,90){\line(0,1){30}}
\put(80,90){\line(3,1){30}}
\put(80,90){\line(2,-3){20}}
\put(80,90){\line(-2,-3){20}}
\put(80,90){\line(-3,1){30}}
\put(60,60){\line(1,0){40}}
\put(60,60){\line(1,-2){20}}
\put(60,60){\line(-2,-3){20}}
\put(60,60){\line(-4,1){40}}
\put(60,60){\line(-1,4){10}}
\put(50,100){\line(-1,-1){30}}
\put(50,100){\line(-3,1){30}}
\put(50,100){\line(-1,4){10}}
\put(50,100){\line(3,2){30}}
\put(80,120){\line(-2,1){40}}
\put(80,120){\line(1,3){10}}
\put(80,120){\line(3,-2){30}}
\put(110,100){\line(-2,5){20}}
\put(110,100){\line(1,1){30}}
\put(110,100){\line(1,-1){30}}
\put(110,100){\line(-1,-4){10}}
\put(100,60){\line(4,1){40}}
\put(100,60){\line(3,-4){30}}
\put(100,60){\line(-1,-2){20}}
\put(130,20){\line(1,5){10}}
\put(130,20){\line(-1,0){50}}
\put(140,130){\line(0,-1){60}}
\put(140,130){\line(-5,2){50}}
\put(40,140){\line(5,1){50}}
\put(40,140){\line(-2,-3){20}}
\put(20,110){\line(0,-4){40}}
\put(40,30){\line(-1,2){20}}
\put(40,30){\line(4,-1){40}}
\put(85,95){$i$}
\put(53,107){$j$}
\put(80,90){\circle*{5}}
\put(60,60){\circle*{5}}
\put(50,100){\circle*{5}}
\put(80,120){\circle*{5}}
\put(110,100){\circle*{5}}
\put(100,60){\circle*{5}}
\put(50,100){\circle{7}}
\put(80,90){\circle{7}}
\put(80,120){\circle{7}}
\put(60,60){\circle{7}}
\put(20,70){\circle{7}}
\put(20,110){\circle{7}}
\put(40,140){\circle{7}}
\end{picture}}

\end{picture}


\end{figure}


The contributions $a_{nm}$ to the system matrix
are therefore different from 0 only in
elements containing node $m$ and the evaluation of the matrix elements
can be performed on an element basis where all coefficients and unknowns
are linear functions of $x$ and $y$.

This approach is straightforward but not very satisfying with the
semi-implicit time stepping scheme for reasons explained below.
Therefore
an other way has been followed in the present formulation. The fluid domain
is still divided in triangles and the water levels are still defined
at the nodes of the grid
and represented by piecewise linear interpolating functions
in the internal of each element, i.e.
\[
        \zeta = \zeta_{m} \phi_{m} \hspace{1cm} m=1,K
\]
However, the transports are now
expanded, over each triangle, with piecewise constant
(non continuous) form functions $\psi_{n}$ over the whole domain. We therefore
write
\[
        U = U_{n} \psi_{n} \hspace{1cm} n=1,J
\]
where $n$ is now running over all
triangles and $J$ is the total number of triangles.
An example of $\psi_{n}$ is given in the lower right part of Fig. 1a.
Note that the form function is constant 1 over the whole element,
but outside the element identically 0. Thus it is discontinuous
at the element borders.

Since we may
identify the center of gravity of the triangle with the point where
the transports $U_{n}$ are defined (contrary to the water levels
$\zeta_{m}$ which are defined on the vertices of the triangles), the
resulting grid may be seen as a staggered grid where the unknowns
are defined on different locations. This kind of grid is usually used
with the finite difference method. With the form functions used here
the grid of the finite element model resembles
very much an Arakawa B-grid that defines the water levels on the center
and the velocities on the four vertices of a square.

Staggered finite elements have been first introduced into
fluid mechanics by Schoenstadt \cite{Schoenstadt80}. 
He showed that the un-staggered
finite element formulation of the shallow water equations has very
poor geostrophic adjustment properties. Williams 
\cite{Williams81a, Williams81b}
proposed a similar algorithm, the one
actually used in this paper, introducing constant form functions for the
velocities. He showed the excellent propagation and geostrophic
adjustment properties of this scheme.


\subsubsection{The Practical Realization}

The integration of the partial differential equation is now performed by
using the subdivision of the domain in elements (triangles). The
water levels $\zeta$ are expanded in piecewise linear functions
$\phi_{m}, \; m=1,K$ and
the transports are expanded in piecewise constant functions
$\psi_{n}, \; n=1,J$ where $K$ and $J$ are the total number of nodes
and elements respectively.

As weighting functions we use $\psi_{n}$ for the momentum equations
and $\phi_{m}$ for the continuity equation. In this way there will
be $K$ equations for the unknowns $\zeta$ (one for each node) and
$J$ equations for the transports (one for each element).

In all cases the consistent mass matrix has been substituted with
the their lumped equivalent. This was mainly done
to avoid solving a linear system in the case of the momentum equations.
But it was of use also in the solution of the continuity equation
because the amount of mass relative to 
one node does not depend on the surrounding
nodes. This was important especially for the flood and dry mechanism
in order to conserve mass.


\subsubsection{Finite Element Equations}

If equations (\ref{un},\ref{vn},\ref{zsys}) are multiplied with their
weighting functions and integrated over an element we can write down
the finite element equations. But the solution of the water levels does
actually not use the continuity equation in the form (\ref{zsys}), but
a slightly different formulation. Starting from equation (\ref{zn}),
multiplied by the weighting function $\Phi_{M}$ and integrated over one
element yields


\[
          \int_{\Omega} \Phi_{N} (\zn-\zo) \: d\Omega \;
+ (\dthalf) \int_{\Omega} 
	  \left( 
	   \Phi_{N} \frac{\dpp (\un + \uo)}{\dpp x} 
+          \Phi_{N} \frac{\dpp (\vn + \vo)}{\dpp y} 
          \right)
	  \: d\Omega \;
= 0
\]
If we integrate by parts the last two integrals we obtain
\[
          \int_{\Omega} \Phi_{N} (\zn-\zo) \: d\Omega \;
- (\dthalf) \int_{\Omega} 
	  \left( 
	    \frac{\dpp \Phi_{N}}{\dpp x} (\un+\uo) 
+           \frac{\dpp \Phi_{N}}{\dpp y} (\vn+\vo)
	  \right)
	  \: d\Omega \;
= 0
\]
plus two line integrals, not shown, over the boundary of each element
that specify the normal flux over the three element
sides. In the interior of the domain,
once all contributions of all elements have
been summed, these terms cancel at every node,
leaving only the contribution of the
line integral on the boundary of the domain. There, however, the
boundary condition to impose is exactly no normal flux over
material boundaries. Thus, the contribution of these line integrals
is zero.

If now the expressions for $\un,\vn$ are introduced, we obtain a system
with again only the water levels as unknowns
\beqa
\int_{\Omega} \Phi_{N} \zn \: d\Omega \;
 & + & (\dt/2)^{2} \alpha g 
\int_{\Omega} H ( \xdif{\Phi_{N}} \xdif{\zn}  \nonumber
 + \ydif{\Phi_{N}} \ydif{\zn} ) \: d\Omega \; \\
 & = &
\int_{\Omega} \Phi_{N} \zo \: d\Omega \;	\nonumber
+ (\dt/2)^{2} \alpha g 
\int_{\Omega} H ( \xdif{\Phi_{N}} \xdif{\zo} 
 + \ydif{\Phi_{N}} \ydif{\zo} ) \: d\Omega \;  \\
 & + &
 (\dt/2)(1+\alpha) \int_{\Omega}  
  ( \xdif{\Phi_{N}} \uo + \ydif{\Phi_{N}} \vo ) \: d\Omega \; \\
 & - & (\dt^{2}/2) \alpha \nonumber
\int_{\Omega} ( \xdif{\Phi_{N}} X + \ydif{\Phi_{N}} Y ) \: d\Omega \; 
\eeqa
Here we have introduced the symbol $\alpha$ as a shortcut for
\[
\alpha = \frac{1}{1+\dt R}
\]
The variables and unknowns may now be expanded with their basis
functions and the complete system may be set up.


%%%%%%%%%%%%%%%%%%%%%%%%%%%%%%%%%%%%%%%%%%%%%%%%%%%%%%%%%%%%%%%%%%%%%%%%%
%%%%%%%%%%%%%%%%%%%%%%%%%%%%%%%%%%%%%%%%%%%%%%%%%%%%%%%%%%%%%%%%%%%%%%%%%
%%%%%%%%%%%%%%%%%%%%%%%%%%%%%%%%%%%%%%%%%%%%%%%%%%%%%%%%%%%%%%%%%%%%%%%%%
%%%%%%%%%%%%%%%%%%%%%%%%%%%%%%%%%%%%%%%%%%%%%%%%%%%%%%%%%%%%%%%%%%%%%%%%%
%%%%%%%%%%%%%%%%%%%%%%%%%%%%%%%%%%%%%%%%%%%%%%%%%%%%%%%%%%%%%%%%%%%%%%%%%

\subsection{Mass Conservation}

It should be pointed out that only through the use of this staggered grid
the semi-implicit time discretization may be implemented in a feasible
manner. If the Galerkin method is applied
 in a naive way to the resulting equation
(\ref{zsys}) (introducing the linear form functions for transports
and water levels and setting up the system matrix),
the model is not mass conserving.
This may be seen in the following way (see Fig. 1b for reference).
In the computation of the water level at
node $i$, only $\zeta$ and transport values
belonging to triangles that contain node $i$ enter the computation
(full circles in Fig. 1b).
But when, in a second step, the barotropic transports
of node $j$ are computed, water levels of nodes that lie further apart
from the original node $i$ are used
(hollow circles in Fig. 1b).
These water levels have not been included in
the computation of $\zeta_i$, the water level at node $i$.
So the computed transports are actually different
from the transports inserted formally in the continuity equation.
The continuity equation is therefore not satisfied.

These contributions of nodes lying further apart could in principle
be accounted for. In this case
not only the triangles
$\Omega_{i}$ around node $i$ but also all the triangles that have
nodes in common with the triangles $\Omega_{i}$ would give
contributions to node $i$, namely all nodes and elements shown
in Fig. 1b.
The result would be
an increase of the bandwidth of the matrix for the $\zeta$ computation
disadvantageous in terms of memory and time requirements.

Using instead the approach of the staggered finite elements, actually
only the water levels of elements around node $i$ are needed for
the computation of the transports in the triangles $\Omega_i$.
In this case the model satisfies the
continuity equation and is perfectly mass conserving.



\subsection{Inter-tidal Flats}

Part of a basin may consist of areas that are
flooded during high tides and emerge as islands at ebb tide. These
inter-tidal flats are quite difficult to handle numerically because
the elements that represent these areas are neither
islands nor water elements. The boundary line defining their
contours is wandering during the evolution
of time and a mathematical model must reproduce this features.

For reasons of computer time savings a simplified algorithm has been chosen
to represent the inter-tidal flats. When the water level in at least
one of the three nodes of an element falls below a minimum value (5 cm)
the element is considered an island and is taken out of the system.
It will be reintroduced only when in all three
nodes the water level is again higher then the minimum value.
Because in dry nodes no water level is computed anymore, an estimate
of the water level has to be given with some sort of extrapolation mechanism
using the water nodes nearby.

This algorithm has the advantage that it is very easy to
implement and very fast. The dynamical features close to the
inter-tidal flats are of course not well reproduced but the
behavior of the method for the rest of the lagoon
gave satisfactory results.

In any case, since the method stores the water levels of the
last time step, before the element is switched off, introducing the
element in a later moment with the same water levels conserves the
mass balance. This method showed a much better performance
than the one where the new elements were introduced with the water
levels taken from the extrapolation of the surrounding nodes.

%%%%%%%%%%%%%%%%%%%%%%%%%%%%%%%%%%%%%%%%%%%%%%%%%%%%%%%%%%%%%%%%%%%%%%%%%
%%%%%%%%%%%%%%%%%%%%%%%%%%%%%%%%%%%%%%%%%%%%%%%%%%%%%%%%%%%%%%%%%%%%%%%%%
%%%%%%%%%%%%%%%%%%%%%%%%%%%%%%%%%%%%%%%%%%%%%%%%%%%%%%%%%%%%%%%%%%%%%%%%%
%%%%%%%%%%%%%%%%%%%%%%%%%%%%%%%%%%%%%%%%%%%%%%%%%%%%%%%%%%%%%%%%%%%%%%%%%
%%%%%%%%%%%%%%%%%%%%%%%%%%%%%%%%%%%%%%%%%%%%%%%%%%%%%%%%%%%%%%%%%%%%%%%%%

