
%------------------------------------------------------------------------
%
%    Copyright (C) 1985-2020  Georg Umgiesser
%
%    This file is part of SHYFEM.
%
%    SHYFEM is free software: you can redistribute it and/or modify
%    it under the terms of the GNU General Public License as published by
%    the Free Software Foundation, either version 3 of the License, or
%    (at your option) any later version.
%
%    SHYFEM is distributed in the hope that it will be useful,
%    but WITHOUT ANY WARRANTY; without even the implied warranty of
%    MERCHANTABILITY or FITNESS FOR A PARTICULAR PURPOSE. See the
%    GNU General Public License for more details.
%
%    You should have received a copy of the GNU General Public License
%    along with SHYFEM. Please see the file COPYING in the main directory.
%    If not, see <http://www.gnu.org/licenses/>.
%
%    Contributions to this file can be found below in the revision log.
%
%------------------------------------------------------------------------

This section explains typical usage of the model. It will show how the
model can be run doing basic 2D simulations, compute T/S, do 3D simulations,
set up the turbulence module etc. This section is conceived as
a simple HOWTO document. For the exact meaning and usage of the single
parameters, please see the section on input parameters.

\subsubsection{2D Hydrodynamic Simulation}

To run a simulation, two things are needed. The first is the description
of the basin and the numerical grid, which must be prepared beforehand
and then must be compiled in a form that the model can use. This is typically
done by the routine |vp| that, starting from a file |.grd| creates a file
|.bas|. This will be called the basin file from now on.

The second thing that is needed is a description of the simulation and
the forcings that have to be applied. This is done through a 
input parameter description file. Here we call it a |STR| file, because
historically these files always ended with an extension of |.str|. However,
any extension can be used.

\begin{figure}
\begin{alltt}
\input{basic.str}
\end{alltt}
\caption{Example of a basic parameter input file ({\tt STR} file)}
\label{fig:str_basic}
\end{figure}

A basic version of an |STR| file can be found in \ref{fig:str_basic}. In
fact, it is so basic, it really does not do anything. Here only the
compulsory parameters have been inserted. These are:

\begin{itemize}

\item An introductory section |$title| where on three lines the following information is given:

\begin{enumerate}
\item A description of the run. This can be any text that fits on one line.
\item The name of the simulation. This name is used for all files that 
the simulation produces. These files differ from each other only by 
their extension.
\item The name of the basin. This is the basin file without the extension
|.ext|.
\end{enumerate}

\item A section |$para| that contains all necessary parameters for the
simulation to be run. The only compulsory parameters are the ones that
specify the start of the simulation |itanf|, its end |itend| and its 
time step |idt|.

\end{itemize}

In order to be more helpful, some more information must be added to the
|STR| file. As an example let's have a look on \ref{fig:str_example}. Here
we have added two parameters that deal with the type of friction
to be used. |ireib| specifies the bottom friction formulation, here
through a simple quadratic bulk formula. (For the exact meaning of the
parameters, please refer to the section lateron where all parameters
are listed.) The parameter |czdef| specifies the value to use for the
bottom drag coefficient.

The lats parameter in the |$para| section is |dragco| which is the
drag coefficient to use for the wind file specified later. If n

ggugguggu

do with 
