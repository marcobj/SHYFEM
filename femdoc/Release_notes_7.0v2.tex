\documentclass[12pt,draft]{article}

\usepackage{amssymb}
\usepackage{amsmath}
%\usepackage{pict2e}
%\usepackage{epic}
\usepackage{shortvrb}
\usepackage{verbatim}

\newcommand{\beq}{\begin{equation}}
\newcommand{\eeq}{\end{equation}}
\newcommand{\beqm}{\begin{multline}}
\newcommand{\eeqm}{\end{multline}}
\newcommand{\shyfem}{SHYFEM}

\newcommand{\Unod}{U}
\newcommand{\Vnod}{V}
\newcommand{\znod}{\zeta}
\newcommand{\Unew}{U^n}
\newcommand{\Vnew}{V^n}
\newcommand{\znew}{\zeta^n}
\newcommand{\Uold}{U}
\newcommand{\Vold}{V}
\newcommand{\zold}{\zeta}
\newcommand{\Udel}{\Delta U}
\newcommand{\Vdel}{\Delta V}
\newcommand{\zdel}{\Delta \zeta}

\newcommand{\UV}{\mathbf{U}}
\newcommand{\UVnew}{\mathbf{U}^n}
\newcommand{\UVdel}{\mathbf{\Delta U}}
\newcommand{\UVhat}{\hat{\mathbf{U}}}
\newcommand{\UVhatdel}{\mathbf{\Delta \hat{U}}}
\newcommand{\HH}{\mathbf{H}}
\newcommand{\Amat}{\mathbf{\bar{A}}}
\newcommand{\AmatI}{\mathbf{\bar{A}}^{-1}}
\newcommand{\FF}{\mathbf{F}}
\newcommand{\II}{\mathbf{I}}
\newcommand{\Ipart}{\mathbf{\partial I}}

\newcommand{\dt}{\Delta t}
\newcommand{\dz}{\Delta z}

\newcommand{\Ftilde}{\tilde{F}}
\newcommand{\Ftx}[1]{\tilde{F}^x_{#1}}
\newcommand{\Fty}[1]{\tilde{F}^y_{#1}}

\newcommand{\delx}{\partial_x}
\newcommand{\dely}{\partial_y}

\newcommand{\dpartt}[1]{\frac{\partial #1}{\partial t}}
\newcommand{\dpartx}[1]{\frac{\partial #1}{\partial x}}
\newcommand{\dparty}[1]{\frac{\partial #1}{\partial y}}
\newcommand{\dpartz}[1]{\frac{\partial #1}{\partial z}}
\newcommand{\dpartzz}[1]{\frac{\partial^2 #1}{\partial z^2}}

\newcounter{cms} % nel preambolo

\MakeShortVerb{\|}

\title{Release notes for \shyfem{} 7.0}
\author{The \shyfem{} Team}

\begin{document}

\maketitle


\section{Overview}


The \shyfem{} version 7 shows major differences with respect to 
the older versions (pre 7). Some of the differences are so
strong that they are not backward compatible with the older versions. This means that changes have to be made to the
input files and also the parameter file. Otherwise version 7
will not be able to run with the old input files. All these changes
are detailed here below.

The fact that version 7 breaks with the compatibility of
the older versions is deliberate. It would have been too complicated
to maintain backward compatibility with older versions. It would
have also made the code extremely complicated.

On the other hand the transition to the new version has been made
quite simple. There exists a utility |bc2fem| that transforms the
old input files to the new file format. The routine is quite 
well documented, so there should really be no problem to
apply this routine to the old files. Changes in the parameter input
file consist in declaring some obligatory parameters and
in the way space varying boundary conditions are specified. All this
is explained in the following section.


\section{New file formats}


In the previous versions different file formats were adopted for 
boundary conditions, initial values,
diagnostic values and meteo forcing. File formats also kept
changing, trying to integrate more information into
the files. At the end of the versions 6 of the model, there were
4 formats for meteo forcing, all incompatible between each other.

Another problem when specifying spatially varying boundary
conditions was the fact that when specifying a spatially
varying parameter at the boundary, all parameters had to
be given spatially varying. It was therefore not possible to 
use plain time series files for this boundary.

To end this situation, in version 7 of the model
it has been decided to standardize
on only two different file formats that would fit all needs. This
file format can now be applied to all types of input, like meteo
forcing, lateral boundary conditions, initial conditions and 
variables used for nudging. 

The two formats are a classical time series format and a more
complicated format that can be used for all the other situations
where simple time series cannot be used anymore. We call in
the following the time series TS and the more 
complete format FEM. 

In the following sections the two input file formats are described
and typical usages are described.

\subsection{The time series (TS) format}

This file format can be used every time when only a simple
series of values have to be imposed. Examples of this are
a time series of water levels to be imposed on one
open boundary, when all values of the open boundary
have the same value. Another example is a spatially
homogeneous wind blowing over the water body. Also
in this case only one value has to be specified.

The format consist in various columns of data. The first column
consists of the time in seconds and the other columns
consist of data. In case of the water level there are two columns,
the first is time, the second is the water level. In case of
the wind there are three columns, their meaning depending on the
parameter |iwtype|. Apart from the first time column, the second and third can be wind velocity in $x,y$ direction, or it can be
wind speed and direction, where the wind speed is in [m/s]
or in nodes.

The format for time series has been kept backward compatible.
Only some minor enhancements have been introduced
that are explained in a separate document, where the exact 
format has been detailed. It is important to note that files
with the time series format from older versions can be used
in version 7 without changes.

\subsection{The finite element (FEM) format}

The FEM file format covers all the other situation where the 
TS format cannot be used. These cases include:
\begin{itemize}
\item spatially varying meteo files,
\item regular (not interpolated) meteo files,
\item spatially varying (2D and 3D) boundary files like
water levels, temperature fields, etc.,
\item 2D and 3D fields for scalar values (temperature, 
salinity) used as initial fields or to be used for nudging.
\end{itemize}
Please note that this list is not complete. 

All of the above usage already was covered in older versions
of \shyfem{}. However, different file formats were adopted 
to fulfill the usage needs. It is important for code
maintenance and readability that all these different formats have been merged into one standard FEM format. This fact 
automatically meant that there had to be an incompatibility
between older versions and version 7.

New FEM files can be directly produced by programs creating the
desired content. The exact FEM file format can be found either
in a different document that only gives details about the format
or in file |subfemfile.f|, the file where the subroutines that write
FEM files have been collected.

Existing old files can be converted by the utility |bc2fem|. The
routine writes some help messages to the screen which should
be easily understandable. For more information the source code
of |bc2fem.f| will be very helpful.

Since the FEM files contain more information than the original 
files, this information has to be provided during the process
of file conversion. The following list shows the extra information contained in the FEM files and explains why it is needed:
\begin{itemize}
\item both formatted and unformatted,
\item an absolute reference date can be given for the file (please see the section on time management),
\item for 3D fields (non vertically homogeneous) the depth
of the nodes and the vertical discretization has to be given.
\end{itemize}

The last point has to be explained a little bit more. \shyfem{}
is now able to vertically interpolate between different vertical
levels, sigma levels included. In order that this is possible the
vertical discretization of the file has to be known. Also the
total depth of every node is needed for the interpolation from
sigma levels, but if this information is not given the total depth
of the node from the model is used.

Various helper routines are available that deal with FEM files. Here
is a probably incomplete list:
\begin{itemize}
\item |bc2fem| - converts old file format to FEM file format
\item |feminf| - information on fem file
\item |femelab| - elaborates (rewrites) FEM files
\item |femformat| - converts between formatted and unformatted FEM files
\end{itemize}



\section{New time management}

Time in older versions of \shyfem{} was given as an integer
with units in seconds. This gave rise to a series of problems,
like
\begin{itemize}
\item there was no absolute reference date 
that would have link the time in seconds to real time.
\item only integer time steps were allowed. Therefore no time step below one second was possible.
\item running time was also measured with integer values. This did not allow simulations much longer than 30 years.
\item it was not clear from a given file to what date 
it was referred to.
\end{itemize}

Because of these deficiencies time management has changed in version 7. Not all of the features given here are already 
implemented, but all features will be finished before version 7.1
will be released.


\subsection{Some definitions}


Here are some definitions that will be used describing
the new time management:
\begin{itemize}
\item date - a unique point in time that is identified by date
(year,month,day) and time (hour,minute,second).
\item time - a time span measured in seconds from some
specific date. To which date this time is referred to depends on how
and where it is used.
\item relative time - this is the time in seconds relative
to a reference date. In \shyfem{} the simulation runs in
relative time referred to the reference date of the simulation.
Start and end of the simulation are specified with
this time. External files have a time stamp that is in relative time
referred to the reference date of the file. If external files have no indication of reference date. 
(like it was in older versions) these files are referred to 
the reference date of the simulation. Units are seconds.
\item absolute time - this is the time in seconds from a unique
reference date, the
first of January of year 1. The fact that this time is referred
always to the same reference date makes it ideal for converting
relative time referred to different reference dates. Units are seconds.
\item reference date - this is the date to which relative time is referred to. There can be more than one reference date. The
simulation has its reference date. Every file can have its own 
reference date. 
\end{itemize}

An example will explain these definitions. Let's say you have 
set the reference date for the simulation at 1.1.2014 (the way
how to do this is given below). In the parameter file the start
of the simulation is given as |itanf=172800|. This is relative time,
relative to the reference date. So the simulation starts at
3.1.2014, midnight. The absolute time for this starting date
is 63524304000, a rather useless number. In fact, you will never get 
into contact with the absolute time. This is all handled in
the program.

If you now use an external file with a reference date set to
2.1.2014. Then the program, in order to pick the first data
to be used for the start of the simulation, will look for a
relative time in the file corresponding to 86400, because the
relative time of 86400 together with the file reference date
of 2.1.2014 corresponds to 3.1.2014, midnight.

In any case, all of this will be handled transparently by the 
program.

\subsection{Changes from older versions}

The first and only incompatibility introduced with version 7 is the 
need to specify the reference date for the simulation. This is
done through the parameters |date| and |time| in the |para|
section of the parameter input file. The format of these parameters is specified below. If you want to continue working with relative
dates just set explicitly |date| to 0. 
In this case nothing will be changed 
with respect to the old versions.

The simulation always runs in relative time. In older versions
there was no need to specify the reference date, so this reference date was implicit. All times in external files had to refer to the same
implicit reference date.

With version 7 the reference date of the simulation has to be specified. The |date| parameter has to be specified as
YYYYMMDD. It is also possible to specify only the year as YYYY. This
is the same as YYYY0101, the first of January of year YYYY. The 
reference time also can be specified in parameter |time|
as hhmmss. A value of 0 means midnight. This is also the
default. Please note that |date| and |time| are integer values,
so an example of specifying these parameters is 
|date|=20141102 and |time|=120000 which indicates as the
reference date the second of November 2014 at noon. It is highly recommended that you leave time always at 0 (midnight). If you
want to set the date to the beginning of the year then 
|date|=2014 is sufficient.

As already explained, if you do not want to set a reference date
(maybe because you are running a simulation not referred to
any external time), please set the |date| parameter explicitly to
0 to indicate your intention. In this case all external files
also have to use this time reference with no reference date
indicated in the files.

\subsection{Reference dates in files}

Once the reference date has been chosen you will have to indicate
all references to time as relative time referred to this reference date. Examples are the start and end of simulation (|itanf| and
|itend|) or the start of the output of hydrodynamic results 
(|itmout|). All time steps (|idt|, |idtout|, etc.) have still to be specified in seconds.

External files can be handled as follows. If the file time is referred
to the same reference date of the simulation nothing has to be
done. \shyfem{} will take care of referring the relative time of the
files to the same reference date of the simulation.

If the external files contain a reference date, then the time
of the data record contained in the file will be converted to
the relative time of the simulation. All of this will happened
without any user interaction. External files can therefore be 
reused for several simulations, even if the reference date
of the simulations change.

\subsubsection{Reference dates in time series (TS files)}

In case of a time series (TS file) a reference date can be easily indicated by simply inserting the following line at the
beginning of the file:
\begin{verbatim}
#date: YYYYMMDD
\end{verbatim}
As already explained, if the date is the first of January, also |YYYY|
will do. These lines can also occur in any other line of the file.
In this case all lines after the |date| line will refer to the new
reference date. This feature makes it possible to simple
concatenate two TS files with different reference date on top
by copying them into one file.


\subsubsection{Reference dates in FEM files}

In case of FEM files two strategies can be followed. The first
one is to fix one reference date for the file and then adjust
the relative time of each record. So, if for example the reference
date is 20140101, a file with hourly data will have relative times
as 0, 3600, 7200 etc..

Another strategy can be to always set the reference date to
the external date of the record and set the relative time to zero.
This works because every data record in FEM files has its own 
reference date. In this case the time record of the aforementioned
hourly file would always be 0, and the reference dates of the
records would read as
\begin{verbatim}
20141010 000000
20141010 010000
20141010 020000
20141010 030000
\end{verbatim}
This solution is probably the preferred version of specifying
time in FEM files because it makes the date and time human
readable.



There is also the possibility to adopt a mixed version of the
above mentioned strategies. However, because of difficult
interpretation of the time records this solution is discouraged.


\subsection{Other changes in time variables}

Still to be finished:
\begin{itemize}
\item all time variables are now double precision. (still to be implemented, tbi)
\item fractional time steps are now allowed. (tbi)
\item initial and final dates can now be given as 
`YYYY-MM-DD::hh:mm:ss'. So |itanf='2014-11-01'| starts the
simulation on the first of November 2014. Time steps still have to
be given in seconds. (tbi)
\end{itemize}



\section{Summary of all incompatible changes}

Here a short summary of all incompatible changes with
regard to older versions is given.

\begin{itemize}
\item file formats have changed, except for time series files. Please use |bc2fem| to convert old files to the new format.
\item some parameters are not needed anymore. They are therefore not accepted anymore by \shyfem. Deleted parameters
are: |nbdim|, |ibfix|
\item external files can now have a reference date to which relative
time is referring. It is highly recommended to use this new feature,
even if it is not compulsory.
\item a reference date for the simulation is now obligatory. If
you do not have such a date (idealized simulations) or do
not want to use it, please set |date| explicitly to 0 in the parameter
input file.
\item name of some executables have changed. Here is a list
of the old and new names:
\begin{itemize}
\item ht - shyfem
\item plotsim - shypost
\item plots - shyplot
\item vp - shypre
\end{itemize}

\end{itemize}







\end{document}

%\begin{bmatrix}
%0 & \cdots & 0 
%\\ \vdots & \ddots & \vdots 
%\\ 0 & \cdots & 0 
%\end{bmatrix}

change formatting of fem file
bc2fem: insist on date (done)
initial values: one or more records
nbdim, ibfix
elim:  - wininf  - scalintp, laplap, scal_intp.  - wincrea, windcrea

 - ht           ---> shyfem
 - vp           ---> shyfem_bas o shyfem_pre
 - feminf       ---> shyfem_file_info
 - femelab      ---> shyfem_file_elab
 - femformat    ---> shyfem_file_format
 - plots        ---> shyfem_plots
 - femrun       ---> shyfem_run
 - basbathy     ---> shyfem_bathy
 - grid         ---> shyfem_grid

