

After the grid creation, with |mesh| or other programs, you must interpolate the
bathymetry.
The bathymetry must be contained in a |grd| file, previously created.
This file, together with the basin onto which the bathymetry has to be 
interpolated, has to be specified for the program |shybas|.
The simplest call is: 

\begin{verbatim}
        shybas -bfile bathy.grd mesh2.grd
\end{verbatim}

where |bathy.grd| is the |grd| file with the bathymetry values and
|mesh2.grd| is the basin for which to interpolate the bathymetry.
Different types of interpolation can be used. Please run
|shybas -h| for more options.


%\subsection{Create basin for FEM model (bandwidth optimization)}

%Before proceeding to the simulations we must first create a
%representation of the basin suitable for the finite element model.
%
%In order to create the finite element reppresentation of the
%grid, please run "vpgrd mesh3". This creates a file mesh3.bas.
%This is a binary file suitable for being read by the finite
%element model.\\

%        vpgrd mesh3\\

