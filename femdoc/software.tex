
%------------------------------------------------------------------------
%
%    Copyright (C) 1985-2018  Georg Umgiesser
%
%    This file is part of SHYFEM.
%
%    SHYFEM is free software: you can redistribute it and/or modify
%    it under the terms of the GNU General Public License as published by
%    the Free Software Foundation, either version 3 of the License, or
%    (at your option) any later version.
%
%    SHYFEM is distributed in the hope that it will be useful,
%    but WITHOUT ANY WARRANTY; without even the implied warranty of
%    MERCHANTABILITY or FITNESS FOR A PARTICULAR PURPOSE. See the
%    GNU General Public License for more details.
%
%    You should have received a copy of the GNU General Public License
%    along with SHYFEM. Please see the file COPYING in the main directory.
%    If not, see <http://www.gnu.org/licenses/>.
%
%    Contributions to this file can be found below in the revision log.
%
%------------------------------------------------------------------------

The source code is composed mainly of Fortran 90 files, but files written
in C, Fortran 77, Perl and Shell scripts are also present.

In order to use the model you have to compile it in a Linux Operating
System. Several software products must be present in order to be able
to compile the model. Please refer to the documentation of your Linux
distribution for installing these programs.

\begin{itemize}

\item The package |make| is required for compilation.

\item The |perl| interpreter and the |bash| shell are necessary for compiling.

\item A Fortran 77 and 90 compiler. Supported compilers are the Gnu 
compiler |gfortran|, the Intel Fortran compiler |ifort| and the Portland 
group |pgf90| Fortran compiler.

\item A C compiler. Supported compilers are the Gnu |gcc|, the Intel C
compiler |icc| or the IBM |xlc| C compiler.

\end{itemize}

Please note that you might already have everything available in your
Linux distribution, with the exception maybe of the Fortran compiler.

To find out what software is installed on your computer and what you
still have to install you can run the following command:

\begin{code}
    make check_software
\end{code}

If you get something like |bash: make: command not found|, then you do
not have make installed. Please first install the |make| command and
then run the command again.

The output of the command will show you what software you will still have
to install. The software is divided into different sections. The first
section is needed software, which you will not be able to do without. The
next section is recommended software, which you really should install,
but for compilation and running you will not necessarily need it. The
last section is software which is optional, but which makes life easier.

You can always run |make check_software| again to check if the software
had been successfully installed. When you are satisfied with the output
you can go to the next section.

Depending on the options that you choose for the compilation you may
need some additional package or library. Usually, the error message
gives you the name of the missing library. The name of the corresponding
package to install can be found at the 
web-page\footnote{https://www.debian.org/distrib/packages} for Debian OS.
Usually, Debian-based (e.g., Ubuntu) distributions have the same name.

Whereas most package names are easy to guess, probably the only problem 
could be the developer X11 libraries. In order to be abel to compile the
program |grid| you will need to install some packages that may have
different names depending on your distribution. The packages you will
have to look for are |libx11-dev|, |x11proto-core-dev| and |libxt-dev|.

Please note that you have to carry out the steps in this section only
the first time you install the model. If you install a new version of
SHYFEM software you can skip these steps.

